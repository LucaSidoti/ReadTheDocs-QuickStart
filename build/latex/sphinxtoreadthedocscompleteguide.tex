%% Generated by Sphinx.
\def\sphinxdocclass{report}
\documentclass[a4paper,10pt,english]{sphinxmanual}
\ifdefined\pdfpxdimen
   \let\sphinxpxdimen\pdfpxdimen\else\newdimen\sphinxpxdimen
\fi \sphinxpxdimen=.75bp\relax
\ifdefined\pdfimageresolution
    \pdfimageresolution= \numexpr \dimexpr1in\relax/\sphinxpxdimen\relax
\fi
%% let collapsible pdf bookmarks panel have high depth per default
\PassOptionsToPackage{bookmarksdepth=5}{hyperref}

\PassOptionsToPackage{booktabs}{sphinx}
\PassOptionsToPackage{colorrows}{sphinx}

\PassOptionsToPackage{warn}{textcomp}
\usepackage[utf8]{inputenc}
\ifdefined\DeclareUnicodeCharacter
% support both utf8 and utf8x syntaxes
  \ifdefined\DeclareUnicodeCharacterAsOptional
    \def\sphinxDUC#1{\DeclareUnicodeCharacter{"#1}}
  \else
    \let\sphinxDUC\DeclareUnicodeCharacter
  \fi
  \sphinxDUC{00A0}{\nobreakspace}
  \sphinxDUC{2500}{\sphinxunichar{2500}}
  \sphinxDUC{2502}{\sphinxunichar{2502}}
  \sphinxDUC{2514}{\sphinxunichar{2514}}
  \sphinxDUC{251C}{\sphinxunichar{251C}}
  \sphinxDUC{2572}{\textbackslash}
\fi
\usepackage{cmap}
\usepackage[T1]{fontenc}
\usepackage{amsmath,amssymb,amstext}
\usepackage{babel}



\usepackage{tgtermes}
\usepackage{tgheros}
\renewcommand{\ttdefault}{txtt}



\usepackage[Bjarne]{fncychap}
\usepackage{sphinx}

\fvset{fontsize=auto}
\usepackage{geometry}

\usepackage{tabenv}

% Include hyperref last.
\usepackage{hyperref}
% Fix anchor placement for figures with captions.
\usepackage{hypcap}% it must be loaded after hyperref.
% Set up styles of URL: it should be placed after hyperref.
\urlstyle{same}

\addto\captionsenglish{\renewcommand{\contentsname}{Contents}}

\usepackage{sphinxmessages}
\setcounter{tocdepth}{1}



\title{Sphinx to Read the Docs: Complete Guide}
\date{Dec 11, 2024}
\release{1.0}
\author{Luca Sidoti}
\newcommand{\sphinxlogo}{\vbox{}}
\renewcommand{\releasename}{Release}
\makeindex
\begin{document}

\ifdefined\shorthandoff
  \ifnum\catcode`\=\string=\active\shorthandoff{=}\fi
  \ifnum\catcode`\"=\active\shorthandoff{"}\fi
\fi

\pagestyle{empty}
\sphinxmaketitle
\pagestyle{plain}
\sphinxtableofcontents
\pagestyle{normal}
\phantomsection\label{\detokenize{index::doc}}


\sphinxAtStartPar
Welcome to the \sphinxstylestrong{Sphinx to Read the Docs} guide! This repository provides a step\sphinxhyphen{}by\sphinxhyphen{}step process for setting up and deploying documentation using Sphinx and Read the Docs, all demonstrated on \sphinxstylestrong{Ubuntu 22.04}. Whether you’re new to Sphinx or looking for a complete workflow from scratch, this guide will walk you through:
\begin{itemize}
\item {} 
\sphinxAtStartPar
Setting up a Sphinx project within a GitHub repository.

\item {} 
\sphinxAtStartPar
Configuring a virtual environment to manage dependencies.

\item {} 
\sphinxAtStartPar
Customizing your documentation using \sphinxstylestrong{reStructuredText} (\sphinxtitleref{.rst}) files.

\item {} 
\sphinxAtStartPar
Enhancing your documentation with themes and useful Sphinx extensions.

\item {} 
\sphinxAtStartPar
Seamlessly deploying your Sphinx project to \sphinxstylestrong{Read the Docs}.

\end{itemize}

\sphinxAtStartPar
For a detailed view of the files and complete project setup, visit the GitHub repository: {[}Sphinx to Read the Docs QuickStart{]}(\sphinxurl{https://github.com/LucaSidoti/ReadTheDocs-QuickStart}).


\chapter{What You’ll Learn}
\label{\detokenize{index:what-youll-learn}}\begin{itemize}
\item {} 
\sphinxAtStartPar
\sphinxstylestrong{Sphinx Setup}: Install Sphinx in a virtual environment and initialize your project.

\item {} 
\sphinxAtStartPar
\sphinxstylestrong{Customization}: Learn to create and format pages using reStructuredText.

\item {} 
\sphinxAtStartPar
\sphinxstylestrong{Deployment}: Configure and deploy your project to Read the Docs with automatic updates.

\end{itemize}

\sphinxstepscope


\section{Setting Up Sphinx}
\label{\detokenize{page_one:setting-up-sphinx}}\label{\detokenize{page_one::doc}}
\sphinxAtStartPar
In this section, we’ll guide you through the process of setting up a Sphinx documentation project from scratch. You will learn how to create a GitHub repository, set up a virtual environment, install Sphinx and its extensions, and configure your project. Follow these steps to get your documentation project up and running.
\begin{enumerate}
\sphinxsetlistlabels{\arabic}{enumi}{enumii}{}{.}%
\item {} 
\sphinxAtStartPar
\sphinxstylestrong{Create a Public GitHub Repository}
\begin{quote}

\sphinxAtStartPar
First, create a new public repository on GitHub to host your Sphinx documentation project.
\end{quote}

\item {} 
\sphinxAtStartPar
\sphinxstylestrong{Clone the Repository Locally}
\begin{quote}

\sphinxAtStartPar
Clone the newly created repository to your local machine using the following command:

\begin{sphinxVerbatim}[commandchars=\\\{\}]
git\PYG{+w}{ }clone\PYG{+w}{ }\PYGZlt{}your\PYGZhy{}repository\PYGZhy{}url\PYGZgt{}
\end{sphinxVerbatim}
\end{quote}

\item {} 
\sphinxAtStartPar
\sphinxstylestrong{Set Up a Virtual Environment}
\begin{quote}

\sphinxAtStartPar
Move into the cloned repository folder:

\begin{sphinxVerbatim}[commandchars=\\\{\}]
\PYG{n+nb}{cd}\PYG{+w}{ }\PYGZlt{}your\PYGZhy{}cloned\PYGZhy{}repository\PYGZgt{}
\end{sphinxVerbatim}

\sphinxAtStartPar
Then set up a virtual environment using either \sphinxtitleref{venv} or \sphinxtitleref{virtualenv}, adapting the command based on your Python version:
\begin{itemize}
\item {} 
\sphinxAtStartPar
To use \sphinxtitleref{venv} (included in the Python standard library):

\end{itemize}

\begin{sphinxVerbatim}[commandchars=\\\{\}]
python3\PYG{+w}{ }\PYGZhy{}m\PYG{+w}{ }venv\PYG{+w}{ }env
\end{sphinxVerbatim}
\begin{itemize}
\item {} 
\sphinxAtStartPar
To use \sphinxtitleref{virtualenv} (requires separate installation):

\end{itemize}

\begin{sphinxVerbatim}[commandchars=\\\{\}]
python3.10\PYG{+w}{ }\PYGZhy{}m\PYG{+w}{ }virtualenv\PYG{+w}{ }env
\end{sphinxVerbatim}
\end{quote}

\item {} 
\sphinxAtStartPar
\sphinxstylestrong{Activate the Virtual Environment}
\begin{quote}

\sphinxAtStartPar
Activate the virtual environment to isolate your project’s dependencies:

\begin{sphinxVerbatim}[commandchars=\\\{\}]
\PYG{n+nb}{source}\PYG{+w}{ }env/bin/activate
\end{sphinxVerbatim}
\end{quote}

\item {} 
\sphinxAtStartPar
\sphinxstylestrong{Install Sphinx}
\begin{quote}

\sphinxAtStartPar
Install Sphinx within your virtual environment:

\begin{sphinxVerbatim}[commandchars=\\\{\}]
pip\PYG{+w}{ }install\PYG{+w}{ }sphinx
\end{sphinxVerbatim}
\end{quote}

\item {} 
\sphinxAtStartPar
\sphinxstylestrong{Install Additional Extensions}
\begin{quote}

\sphinxAtStartPar
Enhance your Sphinx setup with useful extensions:

\begin{sphinxVerbatim}[commandchars=\\\{\}]
pip\PYG{+w}{ }install\PYG{+w}{ }sphinx\PYGZus{}rtd\PYGZus{}theme\PYG{+w}{ }sphinx\PYGZhy{}copybutton\PYG{+w}{ }sphinx\PYGZus{}code\PYGZus{}tabs\PYG{+w}{ }sphinx\PYGZhy{}new\PYGZhy{}tab\PYGZhy{}link
\end{sphinxVerbatim}
\end{quote}

\item {} 
\sphinxAtStartPar
\sphinxstylestrong{Create Sphinx Documentation}
\begin{quote}

\sphinxAtStartPar
Initialize your Sphinx documentation project:

\begin{sphinxVerbatim}[commandchars=\\\{\}]
sphinx\PYGZhy{}quickstart
\end{sphinxVerbatim}

\begin{sphinxadmonition}{tip}{Tip:}
\sphinxAtStartPar
It’s a good practice to separate source and build directories to keep everything organized.
\end{sphinxadmonition}
\end{quote}

\item {} 
\sphinxAtStartPar
\sphinxstylestrong{Open the Project in Visual Studio Code}
\begin{quote}

\sphinxAtStartPar
Launch Visual Studio Code in the project directory:

\begin{sphinxVerbatim}[commandchars=\\\{\}]
code\PYG{+w}{ }.
\end{sphinxVerbatim}
\end{quote}

\item {} 
\sphinxAtStartPar
\sphinxstylestrong{Activate the Virtual Environment in VS Code}
\begin{quote}

\sphinxAtStartPar
Open a terminal in VS Code and activate your virtual environment:

\begin{sphinxVerbatim}[commandchars=\\\{\}]
\PYG{n+nb}{source}\PYG{+w}{ }env/bin/activate
\end{sphinxVerbatim}
\end{quote}

\item {} 
\sphinxAtStartPar
\sphinxstylestrong{Generate HTML Documentation}

\sphinxAtStartPar
Build the HTML documentation to see the output:

\begin{sphinxVerbatim}[commandchars=\\\{\}]
make\PYG{+w}{ }html
\end{sphinxVerbatim}

\begin{sphinxadmonition}{note}{Note:}
\sphinxAtStartPar
The generated \sphinxtitleref{index.html} file will be located in the \sphinxtitleref{build/html} directory.
\end{sphinxadmonition}

\begin{sphinxadmonition}{warning}{Warning:}
\sphinxAtStartPar
To preview the latest version of the HTML page, remember to refresh your browser. If changes do not appear, first ensure that the build succeeded without errors. If the build is successful but changes are still not visible, try running the following commands:

\begin{sphinxVerbatim}[commandchars=\\\{\}]
make\PYG{+w}{ }clean
make\PYG{+w}{ }html
\end{sphinxVerbatim}

\sphinxAtStartPar
Then, open the new \sphinxtitleref{index.html} file in the \sphinxtitleref{build/html} directory to check the updated documentation.
\end{sphinxadmonition}

\item {} 
\sphinxAtStartPar
\sphinxstylestrong{Add Extensions to \textasciigrave{}conf.py\textasciigrave{}}

\sphinxAtStartPar
Edit the \sphinxtitleref{conf.py} file to include the extensions you installed:

\begin{sphinxVerbatim}[commandchars=\\\{\}]
\PYG{n}{extensions} \PYG{o}{=} \PYG{p}{[}\PYG{l+s+s1}{\PYGZsq{}}\PYG{l+s+s1}{sphinx\PYGZus{}rtd\PYGZus{}theme}\PYG{l+s+s1}{\PYGZsq{}}\PYG{p}{,} \PYG{l+s+s1}{\PYGZsq{}}\PYG{l+s+s1}{sphinx\PYGZus{}copybutton}\PYG{l+s+s1}{\PYGZsq{}}\PYG{p}{,} \PYG{l+s+s1}{\PYGZsq{}}\PYG{l+s+s1}{sphinx\PYGZus{}code\PYGZus{}tabs}\PYG{l+s+s1}{\PYGZsq{}}\PYG{p}{,} \PYG{l+s+s1}{\PYGZsq{}}\PYG{l+s+s1}{sphinx\PYGZus{}new\PYGZus{}tab\PYGZus{}link}\PYG{l+s+s1}{\PYGZsq{}}\PYG{p}{,} \PYG{l+s+s1}{\PYGZsq{}}\PYG{l+s+s1}{sphinx\PYGZus{}togglebutton}\PYG{l+s+s1}{\PYGZsq{}}\PYG{p}{]}
\end{sphinxVerbatim}

\item {} 
\sphinxAtStartPar
\sphinxstylestrong{Change the Theme}

\sphinxAtStartPar
Set the theme for your documentation in \sphinxtitleref{conf.py}:

\begin{sphinxVerbatim}[commandchars=\\\{\}]
\PYG{n}{html\PYGZus{}theme} \PYG{o}{=} \PYG{l+s+s1}{\PYGZsq{}}\PYG{l+s+s1}{sphinx\PYGZus{}rtd\PYGZus{}theme}\PYG{l+s+s1}{\PYGZsq{}}
\end{sphinxVerbatim}

\item {} 
\sphinxAtStartPar
\sphinxstylestrong{Configure Theme Options}

\sphinxAtStartPar
Customize the theme options in \sphinxtitleref{conf.py}:

\begin{sphinxVerbatim}[commandchars=\\\{\}]
\PYG{n}{html\PYGZus{}theme\PYGZus{}options} \PYG{o}{=} \PYG{p}{\PYGZob{}}
    \PYG{l+s+s1}{\PYGZsq{}}\PYG{l+s+s1}{logo\PYGZus{}only}\PYG{l+s+s1}{\PYGZsq{}}\PYG{p}{:} \PYG{k+kc}{False}\PYG{p}{,}
    \PYG{l+s+s1}{\PYGZsq{}}\PYG{l+s+s1}{collapse\PYGZus{}navigation}\PYG{l+s+s1}{\PYGZsq{}}\PYG{p}{:} \PYG{k+kc}{True}\PYG{p}{,}
    \PYG{l+s+s1}{\PYGZsq{}}\PYG{l+s+s1}{sticky\PYGZus{}navigation}\PYG{l+s+s1}{\PYGZsq{}}\PYG{p}{:} \PYG{k+kc}{True}\PYG{p}{,}
    \PYG{l+s+s1}{\PYGZsq{}}\PYG{l+s+s1}{includehidden}\PYG{l+s+s1}{\PYGZsq{}}\PYG{p}{:} \PYG{k+kc}{True}\PYG{p}{,}
    \PYG{l+s+s1}{\PYGZsq{}}\PYG{l+s+s1}{navigation\PYGZus{}depth}\PYG{l+s+s1}{\PYGZsq{}}\PYG{p}{:} \PYG{l+m+mi}{4}\PYG{p}{,}
    \PYG{l+s+s1}{\PYGZsq{}}\PYG{l+s+s1}{titles\PYGZus{}only}\PYG{l+s+s1}{\PYGZsq{}}\PYG{p}{:} \PYG{k+kc}{False}
\PYG{p}{\PYGZcb{}}
\end{sphinxVerbatim}

\item {} 
\sphinxAtStartPar
\sphinxstylestrong{Add a New Page}
\begin{enumerate}
\sphinxsetlistlabels{\alph}{enumii}{enumiii}{}{.}%
\item {} 
\sphinxAtStartPar
Create a new \sphinxtitleref{.rst} file for your page in the \sphinxtitleref{source} directory:

\end{enumerate}

\begin{sphinxVerbatim}[commandchars=\\\{\}]
touch\PYG{+w}{ }source/new\PYGZus{}page.rst
\end{sphinxVerbatim}
\begin{enumerate}
\sphinxsetlistlabels{\alph}{enumii}{enumiii}{}{.}%
\setcounter{enumii}{1}
\item {} 
\sphinxAtStartPar
Add content to \sphinxtitleref{new\_page.rst}, starting with a title:

\end{enumerate}

\begin{sphinxVerbatim}[commandchars=\\\{\}]
\PYG{g+gh}{New Page Title}
\PYG{g+gh}{==============}

\PYG{g+gh}{New Page Subtitle}
\PYG{g+gh}{\PYGZhy{}\PYGZhy{}\PYGZhy{}\PYGZhy{}\PYGZhy{}\PYGZhy{}\PYGZhy{}\PYGZhy{}\PYGZhy{}\PYGZhy{}\PYGZhy{}\PYGZhy{}\PYGZhy{}\PYGZhy{}\PYGZhy{}\PYGZhy{}\PYGZhy{}}
\end{sphinxVerbatim}
\begin{enumerate}
\sphinxsetlistlabels{\alph}{enumii}{enumiii}{}{.}%
\setcounter{enumii}{2}
\item {} 
\sphinxAtStartPar
Update the \sphinxtitleref{toctree} directive in \sphinxtitleref{index.rst} to include your new page:

\end{enumerate}

\begin{sphinxVerbatim}[commandchars=\\\{\}]
\PYG{p}{..} \PYG{o+ow}{toctree}\PYG{p}{::}
    \PYG{n+nc}{:maxdepth:} 2
    \PYG{n+nc}{:caption:} Contents

    new\PYGZus{}page.rst
\end{sphinxVerbatim}

\end{enumerate}

\sphinxstepscope


\section{How to Write Documentation with reStructuredText}
\label{\detokenize{page_two:how-to-write-documentation-with-restructuredtext}}\label{\detokenize{page_two::doc}}
\sphinxAtStartPar
This guide will help you understand how to format and structure your documentation using \sphinxstylestrong{reStructuredText (RST)}. RST is the markup language used by Sphinx and other tools to generate clean and organized documentation.

\sphinxAtStartPar
For more details, refer to the official \sphinxhref{https://www.sphinx-doc.org/en/master/usage/restructuredtext/index.html}{reStructuredText documentation}.

\sphinxAtStartPar
—


\subsection{Headings}
\label{\detokenize{page_two:headings}}
\sphinxAtStartPar
reStructuredText supports multiple levels of headings. Here are a few examples:
\begin{itemize}
\item {} 
\sphinxAtStartPar
\sphinxstylestrong{Level 1 Heading} (with \sphinxtitleref{=} symbols below the heading):

\begin{sphinxVerbatim}[commandchars=\\\{\}]
\PYG{g+gh}{This is a Level 1 Heading}
\PYG{g+gh}{=========================}
\end{sphinxVerbatim}

\sphinxAtStartPar
\sphinxstylestrong{Note:} The \sphinxtitleref{=} symbols must be at least as long as the heading text.

\item {} 
\sphinxAtStartPar
\sphinxstylestrong{Level 2 Heading} (with \sphinxtitleref{\sphinxhyphen{}} symbols below the heading):

\begin{sphinxVerbatim}[commandchars=\\\{\}]
\PYG{g+gh}{This is a Level 2 Heading}
\PYG{g+gh}{\PYGZhy{}\PYGZhy{}\PYGZhy{}\PYGZhy{}\PYGZhy{}\PYGZhy{}\PYGZhy{}\PYGZhy{}\PYGZhy{}\PYGZhy{}\PYGZhy{}\PYGZhy{}\PYGZhy{}\PYGZhy{}\PYGZhy{}\PYGZhy{}\PYGZhy{}\PYGZhy{}\PYGZhy{}\PYGZhy{}\PYGZhy{}\PYGZhy{}\PYGZhy{}\PYGZhy{}\PYGZhy{}}
\end{sphinxVerbatim}

\sphinxAtStartPar
\sphinxstylestrong{Note:} The \sphinxtitleref{\sphinxhyphen{}} symbols must be at least as long as the heading text.

\item {} 
\sphinxAtStartPar
\sphinxstylestrong{Level 3 Heading} (with \sphinxtitleref{\textasciitilde{}} symbols below the heading):

\begin{sphinxVerbatim}[commandchars=\\\{\}]
\PYG{g+gh}{This is a Level 3 Heading}
\PYG{g+gh}{\PYGZti{}\PYGZti{}\PYGZti{}\PYGZti{}\PYGZti{}\PYGZti{}\PYGZti{}\PYGZti{}\PYGZti{}\PYGZti{}\PYGZti{}\PYGZti{}\PYGZti{}\PYGZti{}\PYGZti{}\PYGZti{}\PYGZti{}\PYGZti{}\PYGZti{}\PYGZti{}\PYGZti{}\PYGZti{}\PYGZti{}\PYGZti{}\PYGZti{}}
\end{sphinxVerbatim}

\sphinxAtStartPar
\sphinxstylestrong{Note:} The \sphinxtitleref{\textasciitilde{}} symbols must be at least as long as the heading text.

\end{itemize}

\sphinxAtStartPar
—


\subsection{Text Formatting}
\label{\detokenize{page_two:text-formatting}}\begin{itemize}
\item {} 
\sphinxAtStartPar
\sphinxstylestrong{Bold Text}: Use double asterisks \sphinxtitleref{**} around the text.

\begin{sphinxVerbatim}[commandchars=\\\{\}]
\PYG{g+gs}{**This is bold**}
\end{sphinxVerbatim}

\sphinxAtStartPar
\sphinxstylestrong{This is bold}

\item {} 
\sphinxAtStartPar
\sphinxstyleemphasis{Italic Text}: Use single asterisks \sphinxtitleref{*} around the text.

\begin{sphinxVerbatim}[commandchars=\\\{\}]
\PYG{g+ge}{*This is italic*}
\end{sphinxVerbatim}

\sphinxAtStartPar
\sphinxstyleemphasis{This is italic}

\item {} 
\sphinxAtStartPar
\sphinxcode{\sphinxupquote{Monospace Text}}: Use double backticks around code or filenames.

\begin{sphinxVerbatim}[commandchars=\\\{\}]
\PYG{l+s}{``}\PYG{l+s}{code\PYGZus{}block.py}\PYG{l+s}{``}
\end{sphinxVerbatim}

\sphinxAtStartPar
\sphinxcode{\sphinxupquote{code\_block.py}}

\item {} 
\sphinxAtStartPar
Hyperlinks: You can link to external URLs using this syntax:

\begin{sphinxVerbatim}[commandchars=\\\{\}]
\PYG{l+s}{`reStructuredText }\PYG{l+s+si}{\PYGZlt{}https://www.sphinx\PYGZhy{}doc.org/en/master/usage/restructuredtext/index.html\PYGZgt{}}\PYG{l+s}{`\PYGZus{}}
\end{sphinxVerbatim}

\sphinxAtStartPar
\sphinxhref{https://www.sphinx-doc.org/en/master/usage/restructuredtext/index.html}{reStructuredText}

\end{itemize}

\sphinxAtStartPar
—


\subsection{Blocks}
\label{\detokenize{page_two:blocks}}
\sphinxAtStartPar
There are different types of blocks to highlight important information:

\begin{sphinxVerbatim}[commandchars=\\\{\}]
\PYG{p}{..} \PYG{o+ow}{note}\PYG{p}{::}
    This is a note block. Use it to provide additional information.
\end{sphinxVerbatim}

\begin{sphinxadmonition}{note}{Note:}
\sphinxAtStartPar
This is a note block. Use it to provide additional information.
\end{sphinxadmonition}

\begin{sphinxVerbatim}[commandchars=\\\{\}]
\PYG{p}{..} \PYG{o+ow}{warning}\PYG{p}{::}
    \PYG{g+gs}{**Warning**}: This is a warning block. Use it to alert the reader to something important!
\end{sphinxVerbatim}

\begin{sphinxadmonition}{warning}{Warning:}
\sphinxAtStartPar
\sphinxstylestrong{Warning}: This is a warning block. Use it to alert the reader to something important!
\end{sphinxadmonition}

\begin{sphinxVerbatim}[commandchars=\\\{\}]
\PYG{p}{..} \PYG{o+ow}{error}\PYG{p}{::}
    \PYG{g+gs}{**Error**}: This block is for highlighting errors or critical information.
\end{sphinxVerbatim}

\begin{sphinxadmonition}{error}{Error:}
\sphinxAtStartPar
\sphinxstylestrong{Error}: This block is for highlighting errors or critical information.
\end{sphinxadmonition}

\begin{sphinxVerbatim}[commandchars=\\\{\}]
\PYG{p}{..} \PYG{o+ow}{tip}\PYG{p}{::}
    This is a tip block. Use it to provide helpful advice.
\end{sphinxVerbatim}

\begin{sphinxadmonition}{tip}{Tip:}
\sphinxAtStartPar
This is a tip block. Use it to provide helpful advice.
\end{sphinxadmonition}

\begin{sphinxVerbatim}[commandchars=\\\{\}]
\PYG{p}{..} \PYG{o+ow}{important}\PYG{p}{::}
    This is an important block. Use it to emphasize key points.
\end{sphinxVerbatim}

\begin{sphinxadmonition}{important}{Important:}
\sphinxAtStartPar
This is an important block. Use it to emphasize key points.
\end{sphinxadmonition}

\sphinxAtStartPar
—


\subsection{Custom Blocks}
\label{\detokenize{page_two:custom-blocks}}
\sphinxAtStartPar
You can define your own custom blocks to present additional content with icons or visual markers.

\begin{sphinxVerbatim}[commandchars=\\\{\}]
\PYG{p}{..} \PYG{o+ow}{admonition}\PYG{p}{::} Custom Block Title

    This is a custom block. Use it to highlight key sections of your documentation.
\end{sphinxVerbatim}

\begin{sphinxadmonition}{note}{Custom Block Title}

\sphinxAtStartPar
This is a custom block. Use it to highlight key sections of your documentation.
\end{sphinxadmonition}

\sphinxAtStartPar
—


\subsection{Toggle Button}
\label{\detokenize{page_two:toggle-button}}
\sphinxAtStartPar
You can create collapsible blocks using the toggle button for content that you don’t want to reveal immediately.

\begin{sphinxVerbatim}[commandchars=\\\{\}]
\PYG{p}{..} \PYG{o+ow}{admonition}\PYG{p}{::} This will be shown
  \PYG{n+nc}{:class:} hint

\PYG{p}{  ..} \PYG{o+ow}{toggle}\PYG{p}{::} Click to expand

    This content will be hidden until the toggle is clicked. You can add more details here, such as code snippets, images, or additional explanations.
\end{sphinxVerbatim}

\begin{sphinxadmonition}{note}{Hint}

\begin{sphinxuseclass}{toggle}
\sphinxAtStartPar
This content will be hidden until the toggle is clicked. You can add more details here, such as code snippets, images, or additional explanations.

\end{sphinxuseclass}\end{sphinxadmonition}

\begin{sphinxadmonition}{important}{Important:}
\sphinxAtStartPar
A new file called \sphinxstyleemphasis{custom.css} need to be added in \sphinxstyleemphasis{source/\_static} with the following content:

\begin{sphinxVerbatim}[commandchars=\\\{\}]
.hint\PYG{+w}{ }\PYG{o}{\PYGZob{}}
\PYG{+w}{  }border\PYGZhy{}color:\PYG{+w}{ }var\PYG{o}{(}\PYGZhy{}\PYGZhy{}pst\PYGZhy{}color\PYGZhy{}success\PYG{o}{)}\PYG{p}{;}
\PYG{+w}{  }\PYGZgt{}\PYG{+w}{ }.admonition\PYGZhy{}title\PYG{+w}{ }\PYG{o}{\PYGZob{}}
\PYG{+w}{    }\PYG{p}{\PYGZam{}}:before\PYG{+w}{ }\PYG{o}{\PYGZob{}}
\PYG{+w}{      }background\PYGZhy{}color:\PYG{+w}{ }var\PYG{o}{(}\PYGZhy{}\PYGZhy{}pst\PYGZhy{}color\PYGZhy{}success\PYG{o}{)}\PYG{p}{;}
\PYG{+w}{    }\PYG{o}{\PYGZcb{}}

\PYG{+w}{    }\PYG{p}{\PYGZam{}}:after\PYG{+w}{ }\PYG{o}{\PYGZob{}}
\PYG{+w}{      }color:\PYG{+w}{ }var\PYG{o}{(}\PYGZhy{}\PYGZhy{}pst\PYGZhy{}color\PYGZhy{}success\PYG{o}{)}\PYG{p}{;}
\PYG{+w}{      }content:\PYG{+w}{ }var\PYG{o}{(}\PYGZhy{}\PYGZhy{}pst\PYGZhy{}icon\PYGZhy{}admonition\PYGZhy{}hint\PYG{o}{)}\PYG{p}{;}
\PYG{+w}{    }\PYG{o}{\PYGZcb{}}
\PYG{+w}{  }\PYG{o}{\PYGZcb{}}
\PYG{o}{\PYGZcb{}}
\end{sphinxVerbatim}

\sphinxAtStartPar
The \sphinxstyleemphasis{conf.py} need to be updated with \sphinxcode{\sphinxupquote{html\_css\_files = {[}\textquotesingle{}custom.css\textquotesingle{}{]}}}.
\end{sphinxadmonition}

\sphinxAtStartPar
—


\subsection{Tables}
\label{\detokenize{page_two:tables}}
\sphinxAtStartPar
You can create tables in reStructuredText using two common formats:
\begin{enumerate}
\sphinxsetlistlabels{\arabic}{enumi}{enumii}{}{.}%
\item {} 
\sphinxAtStartPar
\sphinxstylestrong{Grid Tables}

\begin{sphinxVerbatim}[commandchars=\\\{\}]
+\PYGZhy{}\PYGZhy{}\PYGZhy{}\PYGZhy{}\PYGZhy{}\PYGZhy{}\PYGZhy{}\PYGZhy{}\PYGZhy{}\PYGZhy{}\PYGZhy{}\PYGZhy{}\PYGZhy{}\PYGZhy{}\PYGZhy{}\PYGZhy{}\PYGZhy{}\PYGZhy{}\PYGZhy{}\PYGZhy{}\PYGZhy{}\PYGZhy{}\PYGZhy{}\PYGZhy{}+\PYGZhy{}\PYGZhy{}\PYGZhy{}\PYGZhy{}\PYGZhy{}\PYGZhy{}\PYGZhy{}\PYGZhy{}\PYGZhy{}\PYGZhy{}\PYGZhy{}\PYGZhy{}+\PYGZhy{}\PYGZhy{}\PYGZhy{}\PYGZhy{}\PYGZhy{}\PYGZhy{}\PYGZhy{}\PYGZhy{}\PYGZhy{}\PYGZhy{}+
\PYG{o}{|} Header row, column 1   | Header 2   | Header 3 |
+========================+============+==========+
\PYG{o}{|} body row 1, column 1   | column 2   | column 3 |
+\PYGZhy{}\PYGZhy{}\PYGZhy{}\PYGZhy{}\PYGZhy{}\PYGZhy{}\PYGZhy{}\PYGZhy{}\PYGZhy{}\PYGZhy{}\PYGZhy{}\PYGZhy{}\PYGZhy{}\PYGZhy{}\PYGZhy{}\PYGZhy{}\PYGZhy{}\PYGZhy{}\PYGZhy{}\PYGZhy{}\PYGZhy{}\PYGZhy{}\PYGZhy{}\PYGZhy{}+\PYGZhy{}\PYGZhy{}\PYGZhy{}\PYGZhy{}\PYGZhy{}\PYGZhy{}\PYGZhy{}\PYGZhy{}\PYGZhy{}\PYGZhy{}\PYGZhy{}\PYGZhy{}+\PYGZhy{}\PYGZhy{}\PYGZhy{}\PYGZhy{}\PYGZhy{}\PYGZhy{}\PYGZhy{}\PYGZhy{}\PYGZhy{}\PYGZhy{}+
\PYG{o}{|} body row 2             | Cells may span        |
+\PYGZhy{}\PYGZhy{}\PYGZhy{}\PYGZhy{}\PYGZhy{}\PYGZhy{}\PYGZhy{}\PYGZhy{}\PYGZhy{}\PYGZhy{}\PYGZhy{}\PYGZhy{}\PYGZhy{}\PYGZhy{}\PYGZhy{}\PYGZhy{}\PYGZhy{}\PYGZhy{}\PYGZhy{}\PYGZhy{}\PYGZhy{}\PYGZhy{}\PYGZhy{}\PYGZhy{}+\PYGZhy{}\PYGZhy{}\PYGZhy{}\PYGZhy{}\PYGZhy{}\PYGZhy{}\PYGZhy{}\PYGZhy{}\PYGZhy{}\PYGZhy{}\PYGZhy{}\PYGZhy{}\PYGZhy{}\PYGZhy{}\PYGZhy{}\PYGZhy{}\PYGZhy{}\PYGZhy{}\PYGZhy{}\PYGZhy{}\PYGZhy{}\PYGZhy{}\PYGZhy{}+
\end{sphinxVerbatim}

\end{enumerate}
\begin{quote}


\begin{savenotes}\sphinxattablestart
\sphinxthistablewithglobalstyle
\centering
\begin{tabulary}{\linewidth}[t]{TTT}
\sphinxtoprule
\sphinxstyletheadfamily 
\sphinxAtStartPar
Header row, column 1
&\sphinxstyletheadfamily 
\sphinxAtStartPar
Header 2
&\sphinxstyletheadfamily 
\sphinxAtStartPar
Header 3
\\
\sphinxmidrule
\sphinxtableatstartofbodyhook
\sphinxAtStartPar
body row 1, column 1
&
\sphinxAtStartPar
column 2
&
\sphinxAtStartPar
column 3
\\
\sphinxhline
\sphinxAtStartPar
body row 2
&\sphinxstartmulticolumn{2}%
\begin{varwidth}[t]{\sphinxcolwidth{2}{3}}
\sphinxAtStartPar
Cells may span
\par
\vskip-\baselineskip\vbox{\hbox{\strut}}\end{varwidth}%
\sphinxstopmulticolumn
\\
\sphinxbottomrule
\end{tabulary}
\sphinxtableafterendhook\par
\sphinxattableend\end{savenotes}
\end{quote}
\begin{enumerate}
\sphinxsetlistlabels{\arabic}{enumi}{enumii}{}{.}%
\setcounter{enumi}{1}
\item {} 
\sphinxAtStartPar
\sphinxstylestrong{Simple Tables}

\begin{sphinxVerbatim}[commandchars=\\\{\}]
====================  ==========  ==========
Header row, column 1  Header 2    Header 3
====================  ==========  ==========
body row 1, column 1  column 2    column 3
body row 2            Cells may span columns
====================  ======================
\end{sphinxVerbatim}

\end{enumerate}
\begin{quote}


\begin{savenotes}\sphinxattablestart
\sphinxthistablewithglobalstyle
\centering
\begin{tabulary}{\linewidth}[t]{TTT}
\sphinxtoprule
\sphinxstyletheadfamily 
\sphinxAtStartPar
Header row, column 1
&\sphinxstyletheadfamily 
\sphinxAtStartPar
Header 2
&\sphinxstyletheadfamily 
\sphinxAtStartPar
Header 3
\\
\sphinxmidrule
\sphinxtableatstartofbodyhook
\sphinxAtStartPar
body row 1, column 1
&
\sphinxAtStartPar
column 2
&
\sphinxAtStartPar
column 3
\\
\sphinxhline
\sphinxAtStartPar
body row 2
&\sphinxstartmulticolumn{2}%
\begin{varwidth}[t]{\sphinxcolwidth{2}{3}}
\sphinxAtStartPar
Cells may span columns
\par
\vskip-\baselineskip\vbox{\hbox{\strut}}\end{varwidth}%
\sphinxstopmulticolumn
\\
\sphinxbottomrule
\end{tabulary}
\sphinxtableafterendhook\par
\sphinxattableend\end{savenotes}
\end{quote}

\sphinxAtStartPar
—


\subsection{Lists}
\label{\detokenize{page_two:lists}}
\sphinxAtStartPar
reStructuredText supports different types of lists, including:
\begin{enumerate}
\sphinxsetlistlabels{\arabic}{enumi}{enumii}{}{.}%
\item {} 
\sphinxAtStartPar
\sphinxstylestrong{Unordered Lists:}

\begin{sphinxVerbatim}[commandchars=\\\{\}]
\PYG{l+m}{*} This is an item
\PYG{l+m}{*} This is another item

  \PYG{l+m}{*} Nested item
  \PYG{l+m}{*} Another nested item
\end{sphinxVerbatim}
\begin{itemize}
\item {} 
\sphinxAtStartPar
This is an item

\item {} 
\sphinxAtStartPar
This is another item

\end{itemize}
\begin{itemize}
\item {} 
\sphinxAtStartPar
Nested item

\item {} 
\sphinxAtStartPar
Another nested item

\end{itemize}

\item {} 
\sphinxAtStartPar
\sphinxstylestrong{Ordered Lists:}

\begin{sphinxVerbatim}[commandchars=\\\{\}]
\PYG{l+m}{1.} First item
\PYG{l+m}{2.} Second item

  \PYG{l+m}{1.} Nested item
  \PYG{l+m}{2.} Another nested item
\end{sphinxVerbatim}
\begin{enumerate}
\sphinxsetlistlabels{\arabic}{enumii}{enumiii}{}{.}%
\item {} 
\sphinxAtStartPar
First item

\item {} 
\sphinxAtStartPar
Second item

\end{enumerate}
\begin{enumerate}
\sphinxsetlistlabels{\arabic}{enumii}{enumiii}{}{.}%
\item {} 
\sphinxAtStartPar
Nested item

\item {} 
\sphinxAtStartPar
Another nested item

\end{enumerate}

\end{enumerate}

\sphinxAtStartPar
—


\subsection{Figures and Images}
\label{\detokenize{page_two:figures-and-images}}
\sphinxAtStartPar
You can include figures or images in your documentation using the following syntax:

\begin{sphinxVerbatim}[commandchars=\\\{\}]
\PYG{p}{..} \PYG{o+ow}{figure}\PYG{p}{::} /path/to/image.png
   \PYG{n+nc}{:align:} center
   \PYG{n+nc}{:width:} 90\PYGZpc{}
   \PYG{n+nc}{:alt:} Image description

   This is an example of a figure with a caption.
\end{sphinxVerbatim}

\begin{figure}[htbp]
\centering
\capstart

\noindent\sphinxincludegraphics[width=0.900\linewidth]{{image1}.jpg}
\caption{This is an example of a figure with a caption.}\label{\detokenize{page_two:id1}}\end{figure}

\sphinxAtStartPar
For regular images without a caption, use the \sphinxtitleref{image} directive:

\begin{sphinxVerbatim}[commandchars=\\\{\}]
\PYG{p}{..} \PYG{o+ow}{image}\PYG{p}{::} /path/to/image.png
   \PYG{n+nc}{:align:} center
   \PYG{n+nc}{:alt:} Example image
\end{sphinxVerbatim}

\noindent{\hspace*{\fill}\sphinxincludegraphics{{image2}.jpg}\hspace*{\fill}}

\sphinxAtStartPar
—


\subsection{Code Blocks}
\label{\detokenize{page_two:code-blocks}}
\sphinxAtStartPar
To include code snippets, use the \sphinxtitleref{code\sphinxhyphen{}block} directive:

\begin{sphinxVerbatim}[commandchars=\\\{\}]
\PYG{p}{..} \PYG{o+ow}{code\PYGZhy{}block}\PYG{p}{::} \PYG{k}{python}

    \PYG{k}{def} \PYG{n+nf}{hello\PYGZus{}world}\PYG{p}{(}\PYG{p}{)}\PYG{p}{:}
        \PYG{n+nb}{print}\PYG{p}{(}\PYG{l+s+s2}{\PYGZdq{}}\PYG{l+s+s2}{Hello, Sphinx!}\PYG{l+s+s2}{\PYGZdq{}}\PYG{p}{)}
\end{sphinxVerbatim}

\begin{sphinxVerbatim}[commandchars=\\\{\}]
\PYG{k}{def} \PYG{n+nf}{hello\PYGZus{}world}\PYG{p}{(}\PYG{p}{)}\PYG{p}{:}
    \PYG{n+nb}{print}\PYG{p}{(}\PYG{l+s+s2}{\PYGZdq{}}\PYG{l+s+s2}{Hello, Sphinx!}\PYG{l+s+s2}{\PYGZdq{}}\PYG{p}{)}
\end{sphinxVerbatim}

\sphinxAtStartPar
For bash commands, specify the language as \sphinxtitleref{bash}:

\begin{sphinxVerbatim}[commandchars=\\\{\}]
\PYG{p}{..} \PYG{o+ow}{code\PYGZhy{}block}\PYG{p}{::} bash

    echo \PYGZdq{}Hello, Sphinx\PYGZdq{}
\end{sphinxVerbatim}

\begin{sphinxVerbatim}[commandchars=\\\{\}]
\PYG{n+nb}{echo}\PYG{+w}{ }\PYG{l+s+s2}{\PYGZdq{}Hello, Sphinx\PYGZdq{}}
\end{sphinxVerbatim}

\sphinxAtStartPar
—


\subsection{Tabs}
\label{\detokenize{page_two:tabs}}
\sphinxAtStartPar
You can organize content into tabs using the \sphinxtitleref{tab} directive from the Sphinx tabs extension:

\begin{sphinxVerbatim}[commandchars=\\\{\}]
\PYG{p}{..} \PYG{o+ow}{tabs}\PYG{p}{::}

\PYG{p}{    ..} \PYG{o+ow}{tab}\PYG{p}{::} Python

\PYG{p}{        ..} \PYG{o+ow}{code\PYGZhy{}block}\PYG{p}{::} \PYG{k}{python}

            \PYG{k}{def} \PYG{n+nf}{example}\PYG{p}{(}\PYG{p}{)}\PYG{p}{:}
                \PYG{k}{return} \PYG{l+s+s2}{\PYGZdq{}}\PYG{l+s+s2}{Python code block}\PYG{l+s+s2}{\PYGZdq{}}

\PYG{p}{    ..} \PYG{o+ow}{tab}\PYG{p}{::} Bash

\PYG{p}{        ..} \PYG{o+ow}{code\PYGZhy{}block}\PYG{p}{::} bash

            echo \PYGZdq{}This is a Bash code block\PYGZdq{}
\end{sphinxVerbatim}

\begin{sphinxuseclass}{tabs}\sphinxSetupCaptionForVerbatim{Python}\begin{tab}
\begin{sphinxVerbatim}[commandchars=\\\{\}]
\PYG{k}{def} \PYG{n+nf}{example}\PYG{p}{(}\PYG{p}{)}\PYG{p}{:}
    \PYG{k}{return} \PYG{l+s+s2}{\PYGZdq{}}\PYG{l+s+s2}{Python code block}\PYG{l+s+s2}{\PYGZdq{}}
\end{sphinxVerbatim}
\end{tab}\sphinxSetupCaptionForVerbatim{Bash}\begin{tab}
\begin{sphinxVerbatim}[commandchars=\\\{\}]
\PYG{n+nb}{echo}\PYG{+w}{ }\PYG{l+s+s2}{\PYGZdq{}This is a Bash code block\PYGZdq{}}
\end{sphinxVerbatim}
\end{tab}
\end{sphinxuseclass}
\sphinxAtStartPar
—


\subsection{Tips and Best Practices}
\label{\detokenize{page_two:tips-and-best-practices}}\begin{itemize}
\item {} 
\sphinxAtStartPar
Use \sphinxstylestrong{RST} consistently to maintain clean documentation structure.

\item {} 
\sphinxAtStartPar
Take advantage of Sphinx extensions for advanced formatting and functionality.

\item {} 
\sphinxAtStartPar
Keep sections modular for easy navigation and readability.

\end{itemize}

\sphinxstepscope


\section{Deploying to Read the Docs}
\label{\detokenize{page_three:deploying-to-read-the-docs}}\label{\detokenize{page_three::doc}}
\sphinxAtStartPar
To publish your documentation on Read the Docs, follow these steps to complete the setup:
\begin{enumerate}
\sphinxsetlistlabels{\arabic}{enumi}{enumii}{}{.}%
\item {} 
\sphinxAtStartPar
\sphinxstylestrong{Add a \textasciigrave{}.readthedocs.yaml\textasciigrave{} File}

\sphinxAtStartPar
Create a \sphinxtitleref{.readthedocs.yaml} file in the root of your project directory (not in the \sphinxtitleref{source} folder). This file specifies the build environment and configuration for Read the Docs.

\begin{sphinxVerbatim}[commandchars=\\\{\}]
\PYG{n+nt}{version}\PYG{p}{:}\PYG{+w}{ }\PYG{l+s}{\PYGZdq{}}\PYG{l+s}{2}\PYG{l+s}{\PYGZdq{}}

\PYG{n+nt}{build}\PYG{p}{:}
\PYG{+w}{  }\PYG{n+nt}{os}\PYG{p}{:}\PYG{+w}{ }\PYG{l+s}{\PYGZdq{}}\PYG{l+s}{ubuntu\PYGZhy{}22.04}\PYG{l+s}{\PYGZdq{}}
\PYG{+w}{  }\PYG{n+nt}{tools}\PYG{p}{:}
\PYG{+w}{    }\PYG{n+nt}{python}\PYG{p}{:}\PYG{+w}{ }\PYG{l+s}{\PYGZdq{}}\PYG{l+s}{3.10}\PYG{l+s}{\PYGZdq{}}

\PYG{n+nt}{python}\PYG{p}{:}
\PYG{+w}{  }\PYG{n+nt}{install}\PYG{p}{:}
\PYG{+w}{    }\PYG{p+pIndicator}{\PYGZhy{}}\PYG{+w}{ }\PYG{n+nt}{requirements}\PYG{p}{:}\PYG{+w}{ }\PYG{l+lScalar+lScalarPlain}{requirements.txt}

\PYG{n+nt}{sphinx}\PYG{p}{:}
\PYG{+w}{  }\PYG{n+nt}{configuration}\PYG{p}{:}\PYG{+w}{ }\PYG{l+lScalar+lScalarPlain}{source/conf.py}

\PYG{c+c1}{\PYGZsh{} formats:}
\PYG{c+c1}{\PYGZsh{}   \PYGZhy{} pdf}
\end{sphinxVerbatim}

\item {} 
\sphinxAtStartPar
\sphinxstylestrong{Create a \textasciigrave{}requirements.txt\textasciigrave{} File}

\sphinxAtStartPar
In the root directory, add a \sphinxtitleref{requirements.txt} file to define the dependencies for your project.

\begin{sphinxVerbatim}[commandchars=\\\{\}]
sphinx\PYGZhy{}rtd\PYGZhy{}theme\PYG{o}{=}\PYG{o}{=}\PYG{l+m}{2}.0.0
sphinx\PYGZhy{}copybutton\PYG{o}{=}\PYG{o}{=}\PYG{l+m}{0}.5.2
sphinx\PYGZhy{}code\PYGZhy{}tabs\PYG{o}{=}\PYG{o}{=}\PYG{l+m}{0}.5.5
sphinx\PYGZhy{}new\PYGZhy{}tab\PYGZhy{}link\PYG{o}{=}\PYG{o}{=}\PYG{l+m}{0}.6.1
sphinx\PYGZhy{}togglebutton\PYG{o}{=}\PYG{o}{=}\PYG{l+m}{0}.3.2
\end{sphinxVerbatim}

\item {} 
\sphinxAtStartPar
\sphinxstylestrong{Push Your Changes to GitHub}

\sphinxAtStartPar
Commit and push all your changes to GitHub:

\begin{sphinxVerbatim}[commandchars=\\\{\}]
\PYG{n+nb}{cd}\PYG{+w}{ }\PYGZlt{}your\PYGZhy{}cloned\PYGZhy{}repository\PYGZgt{}
git\PYG{+w}{ }status
git\PYG{+w}{ }add\PYG{+w}{ }.
git\PYG{+w}{ }status
git\PYG{+w}{ }commit\PYG{+w}{ }\PYGZhy{}m\PYG{+w}{ }\PYG{l+s+s2}{\PYGZdq{}Add Read the Docs configuration\PYGZdq{}}
git\PYG{+w}{ }push
\end{sphinxVerbatim}

\item {} 
\sphinxAtStartPar
\sphinxstylestrong{Login to Read the Docs}

\sphinxAtStartPar
Sign in to Read the Docs using your GitHub account at {[}Read the Docs Login{]}(\sphinxurl{https://readthedocs.org/accounts/login/}).

\item {} 
\sphinxAtStartPar
\sphinxstylestrong{Import Your Project}

\sphinxAtStartPar
Click \sphinxcode{\sphinxupquote{Import a Project}} and select your GitHub repository from the list.

\item {} 
\sphinxAtStartPar
\sphinxstylestrong{Configure Your Project}

\sphinxAtStartPar
Adjust any necessary settings for your project, but you can leave most settings at their default values.

\item {} 
\sphinxAtStartPar
\sphinxstylestrong{Build Your Documentation}

\sphinxAtStartPar
After importing, go to the \sphinxcode{\sphinxupquote{Builds}} section of your project on Read the Docs. If everything is configured correctly, the build process should complete without errors.

\item {} 
\sphinxAtStartPar
\sphinxstylestrong{View Your Documentation}

\sphinxAtStartPar
Once the build is successful, click on \sphinxcode{\sphinxupquote{View Docs}} to see your masterpiece come to life. Share the URL from the Overview tab with your friends, colleagues, or anyone who’ll appreciate your documentation brilliance. Feel free to brag a little — after all, you’ve earned it!

\end{enumerate}

\sphinxAtStartPar
Congratulations, you’ve successfully deployed your documentation to Read the Docs!



\renewcommand{\indexname}{Index}
\printindex
\end{document}